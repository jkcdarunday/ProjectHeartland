\documentclass{sigchi}

% Use this command to override the default ACM copyright statement
% (e.g. for preprints).  Consult the conference website for the
% camera-ready copyright statement.

%% EXAMPLE BEGIN -- HOW TO OVERRIDE THE DEFAULT COPYRIGHT STRIP -- (July 22, 2013 - Paul Baumann)
%\toappear{One does not simply copy this document.}
%% EXAMPLE END -- HOW TO OVERRIDE THE DEFAULT COPYRIGHT STRIP -- (July 22, 2013 - Paul Baumann)

% Arabic page numbers for submission.  Remove this line to eliminate
% page numbers for the camera ready copy
% \pagenumbering{arabic}

% Load basic packages
\usepackage{balance}  % to better equalize the last page
\usepackage{graphics} % for EPS, load graphicx instead 
\usepackage[T1]{fontenc}
\usepackage{txfonts}
\usepackage{mathptmx}
\usepackage[pdftex]{hyperref}
\usepackage{color}
\usepackage{booktabs}
\usepackage{textcomp}
% Some optional stuff you might like/need.
\usepackage{microtype} % Improved Tracking and Kerning
% \usepackage[all]{hypcap}  % Fixes bug in hyperref caption linking
\usepackage{ccicons}  % Cite your images correctly!
\usepackage[utf8]{inputenc} % for a UTF8 editor only

% If you want to use todo notes, marginpars etc. during creation of your draft document, you
% have to enable the "chi_draft" option for the document class. To do this, change the very first
% line to: "\documentclass[chi_draft]{sigchi}". You can then place todo notes by using the "\todo{...}"
% command. Make sure to disable the draft option again before submitting your final document.
\usepackage{todonotes}

% Paper metadata (use plain text, for PDF inclusion and later
% re-using, if desired).  Use \emtpyauthor when submitting for review
% so you remain anonymous.
\def\plaintitle{University of the Philippines Los Baños Online Student Information System}
\def\plainauthor{Jan Keith Darunday, Ferriel Lisandro Melarpis, Toni-Jan Keith Monserrat, and Reginald Neil Recario}
\def\emptyauthor{}
\def\plainkeywords{Authors' choice; of terms; separated; by
  semicolons; include commas, within terms only; required.}
\def\plaingeneralterms{Documentation, Standardization}

% llt: Define a global style for URLs, rather that the default one
\makeatletter
\def\url@leostyle{%
  \@ifundefined{selectfont}{
    \def\UrlFont{\sf}
  }{
    \def\UrlFont{\small\bf\ttfamily}
  }}
\makeatother
\urlstyle{leo}

% To make various LaTeX processors do the right thing with page size.
\def\pprw{8.5in}
\def\pprh{11in}
\special{papersize=\pprw,\pprh}
\setlength{\paperwidth}{\pprw}
\setlength{\paperheight}{\pprh}
\setlength{\pdfpagewidth}{\pprw}
\setlength{\pdfpageheight}{\pprh}

% Make sure hyperref comes last of your loaded packages, to give it a
% fighting chance of not being over-written, since its job is to
% redefine many LaTeX commands.
\definecolor{linkColor}{RGB}{6,125,233}
\hypersetup{%
  pdftitle={\plaintitle},
% Use \plainauthor for final version.
%  pdfauthor={\plainauthor},
  pdfauthor={\emptyauthor},
  pdfkeywords={\plainkeywords},
  bookmarksnumbered,
  pdfstartview={FitH},
  colorlinks,
  citecolor=black,
  filecolor=black,
  linkcolor=black,
  urlcolor=linkColor,
  breaklinks=true,
}

% create a shortcut to typeset table headings
% \newcommand\tabhead[1]{\small\textbf{#1}}


% remove copyright
\makeatletter
\def\@copyrightspace{\relax}
\makeatother

% End of preamble. Here it comes the document.
\begin{document}

\title{\plaintitle}

%Jan Keith Darunday, Ferriel Lisandro Melarpis, Toni-Jan Keith Monserrat, and Reginald Neil Recario
\numberofauthors{4}
\author{%
  \alignauthor{Jan Keith Darunday\\
    \affaddr{Los Ba\~{n}os, Philippines}\\
    \email{jcdarunday@up.edu.ph}}\\
  \alignauthor{Ferriel Lisandro Melarpis\\
    \affaddr{Los Ba\~{n}os, Philippines}\\
    \email{fbmelarpis@up.edu.ph}}\\
  \alignauthor{Toni-Jan Keith Monserrat\\
    \affaddr{Los Ba\~{n}os, Philippines}\\
    \email{tonijanmonserrat@gmail.com}}\\
%   \alignauthor{Reginald Neil Recario\\
%     \affaddr{Los Ba\~{n}os, Philippines}\\
%     \email{rcrecario@up.edu.ph}}\\
}

\maketitle

% \begin{abstract}
% \end{abstract}
% 
% \category{K.3.2}{Computing Milieux}{COMPUTERS AND EDUCATION }{Computer and Information Science Education }
% 
\keywords{student; information; registration; enrollment}

\section{Introduction}

In campuses under the University of the Philippines (UP) System, students have 
more freedom to choose the subjects that they want to enroll in compared to other
universities. UP employs the Revitalized General Education Program (RGEP) that
requires students to take General Education (GE) courses. They have the ability to
choose which of these courses they would like to enlist in.
These GE courses are courses that allow students from any degree program to enroll
unlike major courses that are only available to certain degree programs. The
ability to choose one’s subjects implied that almost all students will need to
manually fill-up their list of courses to enroll to before the beginning of 
every semester. Moreover, courses are divided into sections each of which have 
a limited number of slots for students that it can accommodate. This meant that 
not only will the students need to manually check for conflicts in their 
schedules, they will also need to check if the course that they want to enlist 
in still has slots remaining.

The complexity of this process along with the increasing availability of Internet access to
the public incited the development of computerized Online Registration Systems (ORS). These
systems allowed the enlistment, cancellation, and swapping of slots for courses over the Internet.
These systems also did the checking of conflicting subjects and availability of slots automatically
so that students will not need to do them manually.
In the University of the Philippines Los Baños (UPLB), the Online Registration System that
is used as of the writing of this paper is called SystemOne. This system has undergone various
revisions and improvements as old technology become deprecated by new ones and problems
encountered in previous versions are fixed. The most recent one identifies itself as the third
version and is code-named “Decaf”. However, this system did not prove to be efficient enough
for the student population of UPLB despite several rewrites and revisions.

As a solution, we propose that there needs to be a complete rewrite of 
the system. Because SystemOne is originally written in the PHP programming 
language which has been known to be slow, using a faster programming language 
would be an improvement. SystemOne also has its database connection as its 
bottleneck but it tries to solve this by using memcache as a caching system. 
However, this wasn't efficient enough as queries executed by the backend is too
dynamic which makes the results hard to cache. In order to alleviate the effects
of this bottlneck, we propose that three differnt database/storage systems will
be used. One will be a RAM-based storage system with high-speed read/write that
does not need to support queries, another would be a database system that 
supports fast queries and uses a non-volatile memory storage, the last one would
be a full-fledged relational database system that will be used for persistent
storage.

\section{Related Work}

There is a limited number of papers that specifically discuss SystemOne. These papers
mostly focus on the various modules that provide different kinds of functionality for SystemOne
and the features to be created rather than the speed and feasibility of the use of the system.
Suayan (1997) developed one of the first registration systems used in UPLB called the
UPLB RegisNet. This system was designed to automate the pre-enlistment of subjects for
students.

Meanwhile, the UPLB Academic Information System was created and had support for
modularity. Through its modularity, various modules were created for it. Cruz (2001) created a
Subjects module that gave it the ability to add, edit, and delete subjects from a list of subjects and
Almodiel (2001) developed a module that allowed teachers to add, search, and view the grades of
a student in their respective classes. Enrique (2001) then developed a Schedule of Classes
module that was responsible for managing and viewing the schedule of a subject, a room, or a
teacher.

After a few years, Javier, et al. (2005) developed the first version of UPLB’s Online
Registration System called SystemOne. This system was also extensible through modularity. In
this version, SystemOne used the Java programming language for implementing the Enlistor and
Registrar module, PHP 5 for the Department module and MySQL as the Database Management
System (DBMS).

In 2010, the development of the UPLB SystemOne V2 began. This version also
used the Java programming language and the MySQL DBMS and was also modular. However,
the difference of this new system from the previous system was that even the core functions were
divided into modules and was developed separately.

In 2012, the third and current version of SystemOne was developed by the SystemOne
Development Team. This system, unlike the previous versions that used Java, used the PHP
programming language along with the MySQL DBMS. Like the second version, this system
supports modules and its core functions are also implemented as modules.

There was one study that focused on the speed and response time of SystemOne. Crespo
(2014) wrote a study entitled “Comparing the Response and Rendering Time of PHP-based
against Javascript-based SystemOne”. This study did a comparison on how much faster or slower
SystemOne would be had it been implemented in Javascript using the NodeJS interpreter. Her
study detailed the performance of her own Javascript-based implementation against the
PHP-based implementation. The results of her study showed that the Javascript-based
implementation yielded a 52 percent increase in overall speed.

These results seem to be very promising considering the fact that NodeJS is known to not be
able to take full advantage of a server computer that has multiple cores yet. This shows that
SystemOne could have been faster if the right choices were made.
These studies show a history of Online Registration Systems used in UPLB. It can be seen in
the papers and articles cited that the use of the best and fastest technology has not really been
given consideration.

\section{Methodology}

\subsection{Database}

The system will utilized three database systems. These database systems are as
follows:
\begin{enumerate}
\item High-Speed Read and Write Storage System - This will be shared across all
clusters. This needs to high speed because it will used to store information
that change very fast and is read repeatedly such as the number of slots for
every subject.

\item High-Speed Read and Medium-Speed Write Relational Database System - Also 
shared across all clusters, this will contain data and information that are not
changed often but are read a repeatedly a lot such as the information for subjects
and their dependencies, user accounts, and recommended courses.

\item Persistent Relational Database System - This will be used as the permanent
storage for all data. The other two storage and database systems will synchronize
to this database at specific intervals.
\end{enumerate}



\subsection{Backend}

The backend will be a server that serves a Representational State Transfer (REST)
Application Programming Interface (API) through the Hypertext Transfer Protocol
(HTTP). The system's backend shall have multithreading support. A thread will be 
spawned for every successful HTTP connection. This thread will handle its own
associated connection and wait for a command before processing it and giving a 
reply. If a command is not received or completed after a specified amount of
time, a timeout will occur and the connection will be dropped along with its
corresponding thread being deallocated from the memory. 

\subsubsection{Conflict Checking}

Every time a modification or addition is done on a student's schedule, the 
backend will check for conflicts before executing the specified action. 
To do this, a bitwise AND will be done on the current user's schedule and the
schedule of the subject to be inserted. If the resulting value is not 0 then
an error reply will be sent.

\subsubsection{Cross-Site HTTP Requests}

The system will be designed to support HTTP Requests from specific domains
where the frontend code will be stored through the Access-Control-Allow-Origin header. 
This is to be able to outsource the load added by frontend data being served as 
it will take a significant amount of networking bandwidth to serve the images and
scripts included. This will also allow the frontend to have heavier images and 
scripts without affecting the performance of the main server.


\subsection{Frontend}

The frontend will be a static web application that fires REST requests to the
backend server. The Javascript Object Notation (JSON) will be used to specify parameters
for commands that require sending non-anonymous data while a normal GET request
will be used for reading anonymous data. The frontend will also have an enlistment queue
where the user can setup which subjects he/she would like to enlist and then
send them as a batch to the server. This would lessen the number of connections
that the server will receive lessening the number of threads running in the 
backend.

The design for the frontend will try to follow Google's Material Design 
Specification although not strictly. Along with this, it shall support mobile
platforms by implementing Responsive Design. 



% % REFERENCES FORMAT
% % References must be the same font size as other body text.
% \bibliographystyle{sigchi}
% \bibliography{paper}

%\cleardoublepage

\end{document}

%%% Local Variables:
%%% mode: latex
%%% TeX-master: t
%%% End:

\documentclass{sigchi}

% Use this command to override the default ACM copyright statement
% (e.g. for preprints).  Consult the conference website for the
% camera-ready copyright statement.

%% EXAMPLE BEGIN -- HOW TO OVERRIDE THE DEFAULT COPYRIGHT STRIP -- (July 22, 2013 - Paul Baumann)
%\toappear{One does not simply copy this document.}
%% EXAMPLE END -- HOW TO OVERRIDE THE DEFAULT COPYRIGHT STRIP -- (July 22, 2013 - Paul Baumann)

% Arabic page numbers for submission.  Remove this line to eliminate
% page numbers for the camera ready copy
% \pagenumbering{arabic}

% Load basic packages
\usepackage{balance}  % to better equalize the last page
\usepackage{graphics} % for EPS, load graphicx instead 
\usepackage[T1]{fontenc}
\usepackage{txfonts}
\usepackage{mathptmx}
\usepackage[pdftex]{hyperref}
\usepackage{color}
\usepackage{booktabs}
\usepackage{textcomp}
% Some optional stuff you might like/need.
\usepackage{microtype} % Improved Tracking and Kerning
% \usepackage[all]{hypcap}  % Fixes bug in hyperref caption linking
\usepackage{ccicons}  % Cite your images correctly!
\usepackage[utf8]{inputenc} % for a UTF8 editor only

% If you want to use todo notes, marginpars etc. during creation of your draft document, you
% have to enable the "chi_draft" option for the document class. To do this, change the very first
% line to: "\documentclass[chi_draft]{sigchi}". You can then place todo notes by using the "\todo{...}"
% command. Make sure to disable the draft option again before submitting your final document.
\usepackage{todonotes}

% Paper metadata (use plain text, for PDF inclusion and later
% re-using, if desired).  Use \emtpyauthor when submitting for review
% so you remain anonymous.
\def\plaintitle{University of the Philippines Los Baños Online Student Information System}
\def\plainauthor{Jan Keith Darunday, Ferriel Lisandro Melarpis, Toni-Jan Keith Monserrat, and Reginald Neil Recario}
\def\emptyauthor{}
\def\plainkeywords{Authors' choice; of terms; separated; by
  semicolons; include commas, within terms only; required.}
\def\plaingeneralterms{Documentation, Standardization}

% llt: Define a global style for URLs, rather that the default one
\makeatletter
\def\url@leostyle{%
  \@ifundefined{selectfont}{
    \def\UrlFont{\sf}
  }{
    \def\UrlFont{\small\bf\ttfamily}
  }}
\makeatother
\urlstyle{leo}

% To make various LaTeX processors do the right thing with page size.
\def\pprw{8.5in}
\def\pprh{11in}
\special{papersize=\pprw,\pprh}
\setlength{\paperwidth}{\pprw}
\setlength{\paperheight}{\pprh}
\setlength{\pdfpagewidth}{\pprw}
\setlength{\pdfpageheight}{\pprh}

% Make sure hyperref comes last of your loaded packages, to give it a
% fighting chance of not being over-written, since its job is to
% redefine many LaTeX commands.
\definecolor{linkColor}{RGB}{6,125,233}
\hypersetup{%
  pdftitle={\plaintitle},
% Use \plainauthor for final version.
%  pdfauthor={\plainauthor},
  pdfauthor={\emptyauthor},
  pdfkeywords={\plainkeywords},
  bookmarksnumbered,
  pdfstartview={FitH},
  colorlinks,
  citecolor=black,
  filecolor=black,
  linkcolor=black,
  urlcolor=linkColor,
  breaklinks=true,
}

% create a shortcut to typeset table headings
% \newcommand\tabhead[1]{\small\textbf{#1}}

% End of preamble. Here it comes the document.
\begin{document}

\title{\plaintitle}

%Jan Keith Darunday, Ferriel Lisandro Melarpis, Toni-Jan Keith Monserrat, and Reginald Neil Recario
\numberofauthors{4}
\author{%
  \alignauthor{Jan Keith Darunday\\
    \affaddr{Los Ba\~{n}os, Philippines}\\
    \email{jcdarunday@up.edu.ph}}\\
  \alignauthor{Ferriel Lisandro Melarpis\\
    \affaddr{Los Ba\~{n}os, Philippines}\\
    \email{fbmelarpis@up.edu.ph}}\\
  \alignauthor{Toni-Jan Keith Monserrat\\
    \affaddr{Los Ba\~{n}os, Philippines}\\
    \email{tonijanmonserrat@gmail.com}}\\
%   \alignauthor{Reginald Neil Recario\\
%     \affaddr{Los Ba\~{n}os, Philippines}\\
%     \email{rcrecario@up.edu.ph}}\\
}

\maketitle

% \begin{abstract}
% \end{abstract}
% 
% \category{K.3.2}{Computing Milieux}{COMPUTERS AND EDUCATION }{Computer and Information Science Education }
% 
\keywords{student; information; registration; enrollment}

\section{Introduction}

In campuses under the University of the Philippines (UP) System, students have 
more freedom to choose the subjects that they want to enroll in compared to other
universities. UP employs the Revitalized General Education Program (RGEP) that
requires students to take General Education (GE) courses. They have the ability to
choose which of these courses they would like to enlist in.
These GE courses are courses that allow students from any degree program to enroll
unlike major courses that are only available to certain degree programs. The
ability to choose one’s subjects implied that almost all students will need to
manually fill-up their list of courses to enroll to before the beginning of 
every semester. Moreover, courses are divided into sections each of which have 
a limited number of slots for students that it can accommodate. This meant that 
not only will the students need to manually check for conflicts in their 
schedules, they will also need to check if the course that they want to enlist 
in still has slots remaining.

The complexity of this process along with the increasing availability of Internet access to
the public incited the development of computerized Online Registration Systems (ORS). These
systems allowed the enlistment, cancellation, and swapping of slots for courses over the Internet.
These systems also did the checking of conflicting subjects and availability of slots automatically
so that students will not need to do them manually.
In the University of the Philippines Los Baños (UPLB), the Online Registration System that
is used as of the writing of this paper is called SystemOne. This system has undergone various
revisions and improvements as old technology become deprecated by new ones and problems
encountered in previous versions are fixed. The most recent one identifies itself as the third
version and is code-named “Decaf”. However, this system did not prove to be efficient enough
for the student population of UPLB despite several rewrites and revisions.

As a solution, we propose that there needs to be a complete rewrite of 
the system. Because SystemOne is originally written in the PHP programming 
language which has been known to be slow, using a faster programming language 
would be an improvement. SystemOne also has its database connection as its 
bottleneck but it tries to solve this by using memcache as a caching system. 
However, this wasn't efficient enough as queries executed by the backend is too
dynamic which makes the results hard to cache. In order to alleviate the effects
of this bottlneck, we propose that three differnt database/storage systems will
be used. One will be a RAM-based storage system with high-speed read/write that
does not need to support queries, another would be a database system that 
supports fast queries and uses a non-volatile memory storage, the last one would
be a full-fledged relational database system that will be used for persistent
storage.


% % REFERENCES FORMAT
% % References must be the same font size as other body text.
% \bibliographystyle{sigchi}
% \bibliography{paper}

\cleardoublepage

\end{document}

%%% Local Variables:
%%% mode: latex
%%% TeX-master: t
%%% End:
